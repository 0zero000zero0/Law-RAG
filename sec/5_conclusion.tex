
\section{\heiti 结论}
这项研究旨在解决当前法律判决预测领域的两大难题:传统模型因为“黑箱”操作缺乏解释性,而大型语言模型(LLM)则存在“幻觉”和专业知识不足的问题,导致难以直接应用。为此,我们提出了一种创新的、可解释的司法判决预测方法,该方法融合了法条约束和类案参考。
我们的方法将复杂的判决任务拆解成几个清晰的步骤:首先,利用LLM精确提取判决的核心要素。接着,我们同步从外部知识库中检索权威的法律条文和高度相似的司法判例。最后,LLM会将这些来自不同来源的信息整合起来,生成最终的判决结果。
实验结果充分证明了我们方法的有效性。在罪名和刑期预测任务上,我们的方法表现出色,F1分数分别高达0.7743和0.5525,超越了所有对比模型。与那些同样整合了外部知识库的先进模型相比,我们的方法在罪名预测的F1分数上提升了约3.3\%,而在高度依赖司法实践经验的刑期预测上,更是实现了17.8\%的显著飞跃。这项性能的巨大提升,主要归功于我们框架的两大核心创新:一是引入了法律条文数据库,为模型的推理提供了坚实的法律依据,有效缓解了“幻觉”问题;二是引入了相似案例数据库,让模型能够学习和借鉴海量的司法实践经验,特别是在需要酌情裁量的刑期预测任务上展现出卓越的性能。此外,我们的消融实验也清楚地表明,本研究提出的各个模块都为方法的有效性贡献良多,验证了其设计的合理性。这些成果共同强调了将大型语言模型的强大语言能力、法律条文的规范性以及司法判例的实践性深度融合,是构建精准可靠的智能司法判决系统的关键所在。
