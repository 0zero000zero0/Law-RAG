
\section{\heiti 结论}
本研究针对当前法律判决预测(LJP)领域中,传统模型因其“黑箱”特性而缺乏可解释性,以及LLM因存在“幻觉”和缺乏专业知识而难以直接应用的双重困境,提出了一种基于法条约束与类案融合的可解释司法判决预测方法。该方法将复杂的判决任务分解为一系列逻辑清晰的子任务:首先通过LLM精准提取判决的核心要素,然后并行地从外部知识库中检索权威的法律条文与高度相似的司法判例,最后再由LLM融合这些多源异构信息,生成最终的判决结果。

实验结果有力地证明了本方法的有效性。我们的方法在罪名和刑期预测任务上的F1分数分别达到了\textbf{0.7743}和\textbf{0.5525},在所有对比模型中取得了最优性能。相较于同样结合了外部知识库的先进基准模型,本方法在罪名预测的F1分数上提升了约\textbf{3.3\%},在对司法实践经验有高度依赖的刑期预测上,性能更是实现了\textbf{17.8\%}的显著提升。这一性能突破的核心贡献在于,本框架通过引入法律条文数据库,为模型的推理提供了坚实的法律依据,有效缓解了内容幻觉问题;更关键的是,通过引入相似案例数据库,模型得以借鉴海量的司法实践经验,从而在对酌情裁量有较高要求的刑期预测上表现卓越。这一结果表明,将LLM的强大语言能力与法律条文的规范性、司法判例的实践性进行深度融合,是提升LJP系统准确性和可靠性的关键。