\subsection{\heiti 相关工作}
法律判决预测(Legal Judgment Prediction, LJP)的早期探索,在数据和算力受限的背景下,主要依赖统计学方法。例如,Kort~\cite{kort1957predicting} 通过多因素复合分析,揭示了美国最高法院在处理特定案件时的判决规律。这一时期的研究还包括应用专家系统将法律知识转化为计算机可处理的规则 ~\cite{susskind1986expert}。然而,这些传统方法的核心局限在于其对噪声数据的高度敏感性以及对人工规则的过度依赖。法律文本的复杂性与模糊性,使得规则制定和特征标注异常困难,简单的数学模型难以捕捉司法实践中纷繁复杂的非线性影响因素,从而限制了其预测性能与泛化能力 ~\cite{deng2023syllogistic}。

为解决此问题,研究转向了机器学习与文本挖掘技术~\cite{chen2013text,goncalves2005evaluating}。通过将案件事实作为输入、判决结果作为标签,学者们利用支持向量机(SVM)~\cite{kianmehr2006crime} 或随机森林(Random Forest) ~\cite{sulea2017exploring}等模型,从案情描述中自动学习特征以预测判决。例如,Katz等 ~\cite{sulea2017exploring} 应用随机森林模型,有效提取了影响美国最高法院判决的关键特征。这类方法的优势在于增强了模型对非线性关系的建模能力,并初步实现了特征提取的自动化。但其短板也十分明显:模型性能高度依赖于人工设计的特征工程,不仅耗费大量人力,且难以挖掘文本背后深层次的语义信息,导致其难以被迁移至更广泛的法律场景。

随着深度学习的兴起,LJP研究迎来了新的突破。基于深度神经网络的模型能够自动捕捉文本中更复杂、更抽象的特征~\cite{cheng2025legal,dong2021legal,feng2022legal,jiang2018interpretable}。Luo等 ~\cite{huang2019improved} 提出的模型利用注意力机制动态识别并聚焦于事实描述中最关键的部分,有效关联了案件事实与罪名认定,显著提升了预测准确性。
为了更全面地模拟司法过程,Yue等 ~\cite{yue2021neurjudge} 构建了一个情境感知的多任务学习框架(NeurJudge),通过协同法条预测、罪名预测等多个子任务,让模型学习到任务间的共享信息,从而提升了主任务的性能~\cite{liu2019multi}。
受BERT等模型成功的启发,法律AI领域涌现出如Legal-BERT ~\cite{liu2021robustly,chalkidis2020legal,deepa2021bidirectional,devlin2019bert,fan2022multi}、Lawformer ~\cite{xiao2021lawformer,du2022glm,fei2023lawbench} 等在海量法律语料上预训练的模型。它们通过迁移学习,将从大规模无标注文本中学到的丰富语言知识应用于下游任务,获得了较好的效果~\cite{cui2021pre,houlsby2019parameter,hu2018few}。

尽管深度学习极大地推动了LJP的发展,但即便是针对法律领域优化的PLM,其能力在与新兴的大语言模型(LLM)的对比中仍显不足。

当前,以LLM为核心的技术范式成为研究热点。其强大的常识推理和零样本/少样本学习能力为LJP带来了新的可能性~\cite{brown2020language,huang2022towards}。将LLM应用于严肃的法律领域仍面临严峻挑战。
未经充分优化的通用LLM在处理法律问题时,容易出现“幻觉”(Hallucination)~\cite{cui2023survey},例如捏造不存在的法律条文、引用错误的案例,或给出超出法定范围的量刑建议~\cite{lewis2020retrieval}。因此,为了解决LLM的幻觉问题,主要优化思路主要分为两条路径:一是通过提示工程(Prompt Engineering)引导模型,例如利用“思维链”(Chain of Thought)~\cite{kojima2022large,izacard2021leveraging}或司法三段论~\cite{huang2023lawyer}来规范其推理逻辑。然而,为特定任务精心设计的提示词(Prompt)往往缺乏通用性,难以直接迁移至其他法律任务,这限制了该方法的可扩展性。
;二是以法律专业数据对LLM进行微调,如Lawyer LLaMA ~\cite{chen2020recall},使其具备更强的“法律素养”。尽管微调是提升专业性的有效途径~\cite{hu2021lora,hu2022lora},但它代价高昂。首先,微调的成功依赖于高质量标注数据和较大的计算资源。其次,微调后的LLM知识体系是静态的,其知识停留在训练数据集的时间点,无法实时跟进法律法规的更新与司法解释的演进~\cite{li2021prefix}。最后,微调过程还可能导致模型对其通用知识的“灾难性遗忘”~\cite{chen2020recall},损害其基础推理能力。

因此本研究提出基于大语言模型,法律引导的的案例融合方法。将提示词工程与检索增强技术,实现司法智能审判的可信判决。该方法通过特定的提示词提取案例描述的关键判决核心要素,保留司法审判理论的核心要素,去除冗余的噪声。 此外,通过引入案例数据库和法律条文数据库,为LLM提供精确的罪名定义和司法实践案例, 综合理论知识与实践经验。最后利用多源异构信息进行综合分析与推理,做出最后的司法判决。该方法无需微调LLM,并且可以通过引入新的法律条文和案例,提供最新的法理知识,避免因为LLM缺少相关知识而产生幻觉。