\zihao{5}{
{\noindent\bf Abstract}\quad
%英文摘要
\zihao{5}{\noindent Legal Judgment Prediction (LJP) is a critical research area for advancing judicial intelligence. However, existing methods face significant challenges. Traditional models are often criticized for their "black-box" nature and inability to handle complex legal logic, while the direct application of Large Language Models (LLM) is plagued by inherent flaws such as content "hallucinations" and a lack of domain-specific legal knowledge. To address these issues, this paper proposes a law-guided case fusion judgment method. This approach decomposes the LJP task into several logical sub-tasks: first, it leverages an LLM to extract core adjudicative elements; second, it employs a Retrieval-Augmented Generation (RAG) mechanism to concurrently retrieve authoritative legal statutes and highly similar precedents; finally, it inputs this multi-source information into an LLM for comprehensive reasoning to generate a structured judgment.
This research aims to systematically improve prediction quality by providing the LLM with explicit legal grounding and practical judicial references. Experimental results demonstrate that our method achieves F1-scores of 0.7743 and 0.5525 on charge and sentencing prediction tasks, respectively. Compared to an advanced baseline model that also utilizes RAG, our approach shows improvements of 2.46 and 8.34 percentage points. The performance gain is particularly significant in sentencing prediction, a task that heavily relies on judicial practice. This validates that our method, by effectively fusing multi-source heterogeneous knowledge, can significantly enhance the accuracy and interpretability of LJP, offering a novel solution for building more reliable and transparent intelligent judicial systems.
\par}}

\vspace {5mm}

\zihao{5-}{\noindent
{\heiti Key words \quad }{Large Language Models; Retrieval-Augmention Generation; Judgment Prediction; Smart Courts;}
}\par\noindent

\zihao{5}
\vskip 1mm