
\section{\heiti 结论}
本研究针对法律判决预测中传统模型的“黑箱”问题和大型语言模型(LLM)的“幻觉”与专业知识不足问题,提出了一种融合法条约束与类案参考的可解释性预测方法。该方法通过LLM提取案件要素,结合从外部知识库检索的法律条文和相似案例,最终生成判决结果。

实验证明,本方法在罪名和刑期预测上表现优异(F1分数分别为0.7743和0.5525),显著优于现有模型,尤其在刑期预测上提升了17.8\%。这主要得益于法律条文的引入有效缓解了“幻觉”问题,而相似案例的参考则为模型提供了丰富的司法实践经验。研究证实,将LLM的语言能力、法条的规范性及判例的实践性相结合,是构建精准、可靠的智能司法判决系统的关键