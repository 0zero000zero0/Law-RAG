\documentclass[10.5pt,compsoc,UTF8]{CjC}
\usepackage{CTEX}
\usepackage{graphicx}
\usepackage{footmisc}
\usepackage{subfigure}
\usepackage{url}
\usepackage{multirow}
\usepackage{multicol}
\usepackage[noadjust]{cite}
\usepackage{amsmath,amsthm}
\usepackage{amssymb,amsfonts}
\usepackage{booktabs}
\usepackage{color}
\usepackage{ccaption}
\usepackage{booktabs}
\usepackage{float}
\usepackage{fancyhdr}
\usepackage{caption}
\usepackage{xcolor,stfloats}
\usepackage{comment}
\setcounter{page}{1}
\graphicspath{{figures/}}
\usepackage{cuted}
\usepackage{captionhack}
\usepackage{epstopdf}
\usepackage{gbt7714}
\usepackage{stfloats}
\usepackage{amsmath}
\usepackage{arydshln}
%===============================%

%=======设置奇偶页页眉=======%
\headevenname{\mbox{\quad} \hfill  \mbox{\zihao{-5}{计\quad \quad 算\quad \quad 机\quad \quad 学\quad \quad 报} \hspace {50mm} \mbox{2023 年}}}%
\headoddname{? 期 \hfill
作者姓名等:论文题目}
%=======设置奇偶页页眉=======%

\renewcommand{\thefootnote}{\fnsymbol{footnote}}
\setcounter{footnote}{0}
\renewcommand\footnotelayout{\zihao{5-}}
\newtheoremstyle{mystyle}{0pt}{0pt}{\normalfont}{1em}{\bf}{}{1em}{}
\theoremstyle{mystyle}
\renewcommand\figurename{figure~}
\renewcommand{\thesubfigure}{(\alph{subfigure})}
\newcommand{\upcite}[1]{\textsuperscript{~\cite{#1}}}
\renewcommand{\labelenumi}{(\arabic{enumi})}
\newcommand{\tabincell}[2]{\begin{tabular}{@{}#1@{}}#2\end{tabular}}
\newcommand{\abc}{\color{white}\vrule width 2pt}
\renewcommand{\bibsection}{}
\makeatletter
\renewcommand{\@biblabel}[1]{[#1]\hfill}
\makeatother
\setlength\parindent{2em}

\pagestyle{CjCheadings}
\setmainfont{Times New Roman}

%===========================================================%
\begin{document}

\hyphenpenalty=50000
\makeatletter
\newcommand\mysmall{\@setfontsize\mysmall{7}{9.5}}
\newenvironment{tablehere}
  {\def\@captype{table}}

\let\temp\footnote
\renewcommand \footnote[1]{\temp{\zihao{-5}#1}}

%=======设置首页页眉=======%
\thispagestyle{empty}%
\begin{table*}[!t]
\vspace {-13mm}
\onecolumn
\noindent\begin{tabular}{p{168mm}}
\zihao{5-}
第??卷\quad 第?期 \hfill 计\quad 算\quad 机\quad 学\quad 报\hfill Vol. ??  No. ?\\
\zihao{5-}
20??年?月 \hfill CHINESE JOURNAL OF COMPUTERS \hfill ???. 20??\\
\hline\\[-5.5mm]
\hline\end{tabular}\end{table*}
%=======设置首页页眉=======%

%中文标题、作者与注脚
{
\centering
\vspace {11mm}
{\zihao{2} \heiti 基于大语言模型, 法律引导的的案例融合方法, 用于司法判决预测 }

\vskip 5mm

{\zihao{3}\fangsong 作者名$^{1)}$\quad  作者名$^{2),3)}$ \quad 作者名$^{3) }$($^*$字体为3号仿宋*作者)}
\footnote{\noindent \zihao{6}\songti 收稿日期:\quad \quad -\quad -\quad ;最终修改稿收到日期:\quad \quad -\quad -\quad .*投稿时不填写此项*. 本课题得到… …基金中文完整名称(No.项目号)、… …基金中文完整名称(No.项目号)、… … 基金中文完整名称(No.项目号)资助.\textsf{作者名1(通信作者)},性别,xxxx年生,学位(或目前学历),职称,是/否计算机学会(CCF)会员(提供会员号),主要研究领域为*****、****.E-mail: **************.\textsf{作者名2(通信作者)},性别,xxxx年生,学位(或目前学历),职称,是/否计算机学会(CCF)会员(提供会员号),主要研究领域为*****、****.E-mail: **************. \textsf{作者名3(通信作者)},性别,xxxx年生,学位(或目前学历),职称,是/否计算机学会(CCF)会员(提供会员号),主要研究领域为*****、****.E-mail: **************.(给出的电子邮件地址应不会因出国、毕业、更换工作单位等原因而变动。请给出所有作者的电子邮件)
第1作者手机号码(投稿时必须提供,以便紧急联系,发表时会删除): … …, E-mail: … …*此部分6号宋体*}

\vspace {1mm}

\zihao{6}{$^{1)}$(单位全名 部门(系)全名, 市(或直辖市) 国家名 邮政编码) *字体为6号宋体*单位}

\zihao{6}{$^{2)}$(单位全名 部门(系)全名, 市(或直辖市) 国家名 邮政编码)*中英文单位名称、作者姓名须一致*}

\zihao{6}{$^{3)}$(单位全名 部门(系)全名, 市(或直辖市) 国家名 邮政编码)}

\zihao{6}{\textsf{论文定稿后,作者署名、单位无特殊情况不能变更。若变更,须提交签章申请,国家为中国可以不写,省会城市不写省名,其他国家必须写国家名。}}
}%中文标题、作者与注脚

\vskip 5mm

\zihao{5-}{
\setlength{\baselineskip}{16pt}\selectfont{
\noindent {\heiti 摘\quad 要\quad }
*中文摘要内容置于此处(英文摘要中要有这些内容),字体为小5号宋体。摘要贡献部分,要有数据支持,不要出现``...大大提高''、``...显著改善''等描述,正确的描述是``比{\ldots}提高X{\%}''、``在{\ldots}上改善X{\%}''。*摘要\par}}

\vspace {5mm}

\zihao{5-}{\noindent
{\heiti 关键词 \quad }{*关键词(中文关键字与英文关键字对应且一致,应有5-7个关键词);关键词;关键词;关键词*  }
}\par\noindent
\zihao{5-}{\heiti 中图法分类号\quad } TP\rm{\quad \quad \quad     }
{\heiti DOI号:\quad } *投稿时不提供DOI号

\vskip 5mm

\begin{center}
\zihao{3}{ \heiti Title *(中英文题目一致)字体为4号Times New Roman,加粗* Title}\\
\vspace {5mm}
\zihao{5}{ NAME Name-Name$^{1)}$ NAME Name$^{2)}$ NAME Name-Name$^{3)}$ *字体为5号Times new Roman*Name}\\
\vspace {1mm}
\zihao{6}{{$^{1)}$(Department of ****, University, City ZipCode, China) *字体为6号Times new Roman* Depart.Correspond}}

\zihao{6}{{$^{2)}$(Department of ****, University, City ZipCode)*中国不写国家名*}}

\zihao{6}{{$^{3)}$(Department of ****, University, City ZipCode, country)*外国写国家名*}}

\end{center}

\begin{center}
	\zihao{3}{ \heiti Explainable Judicial Outcome Prediction: A Legal Provision-Constrained and Case-Based Fusion Framework}\\
	% \vspace {5mm}
	% \zihao{5}{ Dingwen Zhang$^{1,\dagger}$\quad Zhentao Liang$^{1,\dagger}$\quad Yili Zhu$^{1}$\quad Hua Zhang$^{1}$\quad Yongbin Qin$^{1,2,3}$\quad  Ruizhang Huang$^{1,2,3}$}

    % \vspace {1mm}

	% \zihao{6}{{$^{1}$Department of Computer and Science, Guizhou University, Guiyang, 550004}}

	% \zihao{6}{{$^{2}$State Key Laboratory of Public Big Data (Guizhou University), Guiyang, 550004}}

	% \zihao{6}{{$^{3}$Text Computing \& Cognitive Intelligence Engineering Research Center of National Education Ministry, Guizhou University, Guiyang, 550004}}
\end{center}

\zihao{5}{
	{\noindent\bf Abstract}\quad
	%英文摘要
	\zihao{5}{\noindent
	Legal judgment prediction is a critical research area driving judicial intelligence. However, current Large Language Models (LLMs) face significant challenges in this task. Without specialized legal knowledge enhancement, LLMs primarily rely on their inherent knowledge for reasoning, which often leads to "hallucinations"—generating judgments inconsistent with legal facts or logic—when handling complex legal cases. Furthermore, LLMs' insufficient deep understanding and precise application of professional legal knowledge severely limit their reliability and interpretability in judicial practice. Legal judgment prediction is a critical research area driving judicial intelligence. However, current Large Language Models (LLMs) face significant challenges in this task. Without specialized legal knowledge enhancement, LLMs primarily rely on their inherent knowledge for reasoning, which often leads to "hallucinations"—generating judgments inconsistent with legal facts or logic—when handling complex legal cases. Furthermore, LLMs' insufficient deep understanding and precise application of professional legal knowledge severely limit their reliability and interpretability in judicial practice. This study aims to systematically improve judgment quality by providing LLMs with clear legal grounds and practical judicial references. Experimental results demonstrate that this method achieved F1 scores of 0.7743 and 0.5525 on the charge prediction and sentence prediction tasks, respectively. Compared to advanced baseline models that also attempted to integrate external knowledge bases, its F1 scores improved by 2.46\% and 8.34\%, respectively. This fully validates that our method, through the effective fusion of multi-source heterogeneous knowledge, can significantly enhance the accuracy of judgment prediction.
	}}

\vspace {5mm}

\zihao{5-}{\noindent
	{\heiti Key words\quad}{Large Language Models; Retrieval-Augmention Generation; Judgment Prediction; Smart Courts;}
}\par\noindent

\zihao{5}
\vskip 1mm

\begin{multicols}{1}
\section{\heiti 引言}

随着信息技术的飞速发展,“数字法院”的建设已成为全球司法领域现代化的重要趋势,对司法审判智能化解决方案的需求日益迫切,尤其在刑事案件审判领域,其核心目标在于提升司法判决的准确性、一致性与效率~\cite{aletras2016predicting}。在这一背景下,法律判决预测(Legal Judgment Prediction, LJP)作为法律人工智能领域的一项基础性、关键性研究任务,受到了学术界与实务界的广泛关注。LJP旨在通过分析案件的事实描述,自动预测法院可能的判决结果,从而辅助法官及其他法律从业者,提高案件处理效率.

LJP技术早期研究主要依赖于人工构建的规则系统和传统的统计机器学习方法~\cite{katz2017general,keown1980mathematical}, 例如支持向量机支持向量机(Support Vector  Machine , SVM)~\cite{boella2011using,kim2015legal}。Sulea~\cite{sulea2017exploring}构建了一种结合多个SVM分类器输出的平均概率集成系统,模型以案情事实描述和时间跨度信息作为输入,能够输出判决结果、法律范围、估算判决日期等信息。Katz[21]使用随机森林,从案情描述中提取有效特征对美国最高法院的判决结果进行预测。但这些方法尚不能挖掘深层的文本特征,且因人工设计的特性,需要大量的人力成本,无法深入应用到其他领域。
\begin{figure*}[htbp]
	\centering
	\includegraphics[width=1\textwidth]{fig/motivation.pdf}
	\caption{动机}
	\label{fig:motivation}
\end{figure*}
然而,传统的自然语言处理和机器学习模型在应用于LJP任务时,往往难以充分捕捉法律文本中复杂的逻辑依赖关系,也难以清晰地呈现法律推理过程~\cite{lin2012exploiting,liu2004case}。
此外,许多早期的深度学习LJP模型如同“黑箱”般运作,其决策过程缺乏透明度和可解释性。不可解释性构成了模型在司法实践中推广应用的主要障碍\cite{ling2017program,ma2021law,nye2021show}。法官难以干预模型的审判逻辑或理解其预测依据,削弱了模型的实用价值。缺乏可解释性还可能引发伦理问题。尤其是模型从历史数据中学习到潜在的偏见,可能导致不公正或不一致的判决结果~\cite{luo2017learning,lv2022improving}。
大型语言模型(LLM)虽因其卓越的语言理解能力而备受期待~\cite{jiang2023legal},但在法律领域的直接应用暴露出若干固有缺陷。
一个核心问题是LLM倾向于产生“幻觉”(Hallucinations)~\cite{lewis2020retrieval,zheng2021when},即生成与客观事实或用户输入不符的内容 。在法律语境下,这可能表现为援引虚假的判例、引言或内部引证,其后果不堪设想。研究表明,在处理特定法律查询时,LLM的幻觉率可能高达69\%至88\%~\cite{Dahl_2024},这种现象往往源于模型在缺乏可验证法律依据的情况下尝试进行推理或生成信息\cite{zhong2020iteratively,zhong2020jec-qa}。

为了解决传统方法在可解释性和处理复杂法律逻辑方面存在不足,而直接应用通用大型语言模型则面临幻觉、缺乏法律知识基础和专业推理能力等问题。因此,本研究提出法律指导的类案融合判决方法,通过提取判决核心要素——犯罪核心要素与证据核心要素,为判决减少干扰因素,提供准确和核心的信息。通过引入法律条文数据库, 为模型提供明确的权威的法律依据,弥补其法律知识的不足,并减少判决“幻觉”现象。 此外,本研究还引入相似案例,旨在为LLM提供司法实践层面的参考,使其理解法律条文在具体情境下的应用方式,学习裁判经验。最后LLM作为核心推理引擎,对这些多源异构信息进行综合分析与推理,综合考量法律原则、司法解释及类案判例的指导作用,输出结构化的判决结果。本研究的方法无需对整个大模型进行重新训练,将复杂的LJP任务分解多个子任务,使得整个推理过程更为透明和模块化。与一些端到端的黑箱模型相比,这种设计不仅有助于提升整体预测的鲁棒性,也为理解和调试模型行为提供了便利。


\subsection{\heiti 相关工作}
法律判决预测(Legal Judgment Prediction, LJP)的早期探索,在数据和算力受限的背景下,主要依赖统计学方法。例如,Kort~\cite{kort1957predicting} 通过多因素复合分析,揭示了美国最高法院在处理特定案件时的判决规律。这一时期的研究还包括应用专家系统将法律知识转化为计算机可处理的规则 ~\cite{susskind1986expert}。然而,这些传统方法的核心局限在于其对噪声数据的高度敏感性以及对人工规则的过度依赖。法律文本的复杂性与模糊性,使得规则制定和特征标注异常困难,简单的数学模型难以捕捉司法实践中纷繁复杂的非线性影响因素,从而限制了其预测性能与泛化能力 ~\cite{deng2023syllogistic,deng2023syllogistic}。

为解决此问题,研究转向了机器学习与文本挖掘技术~\cite{chen2013text,goncalves2005evaluating}。通过将案件事实作为输入、判决结果作为标签,学者们利用支持向量机(SVM)~\cite{kianmehr2006crime} 或随机森林(Random Forest) ~\cite{sulea2017exploring}等模型,从案情描述中自动学习特征以预测判决。例如,Katz等 ~\cite{sulea2017exploring} 应用随机森林模型,有效提取了影响美国最高法院判决的关键特征。这类方法的优势在于增强了模型对非线性关系的建模能力,并初步实现了特征提取的自动化。但其短板也十分明显:模型性能高度依赖于人工设计的特征工程,不仅耗费大量人力,且难以挖掘文本背后深层次的语义信息,导致其难以被迁移至更广泛的法律场景。

随着深度学习的兴起,LJP研究迎来了新的突破。基于深度神经网络的模型能够自动捕捉文本中更复杂、更抽象的特征~\cite{cheng2025legal,dong2021legal,feng2022legal,jiang2018interpretable}。Luo等 ~\cite{huang2019improved} 提出的模型利用注意力机制动态识别并聚焦于事实描述中最关键的部分,有效关联了案件事实与罪名认定,显著提升了预测准确性。
为了更全面地模拟司法过程,Yue等 ~\cite{yue2021neurjudge} 构建了一个情境感知的多任务学习框架(NeurJudge),通过协同法条预测、罪名预测等多个子任务,让模型学习到任务间的共享信息,从而提升了主任务的性能~\cite{liu2019multi}。
受BERT等模型成功的启发,法律AI领域涌现出如Legal-BERT ~\cite{liu2021robustly,chalkidis2020legal,deepa2021bidirectional,devlin2019bert,fan2022multi}、Lawformer ~\cite{xiao2021lawformer,du2022glm,fei2023lawbench} 等在海量法律语料上预训练的模型。它们通过迁移学习,将从大规模无标注文本中学到的丰富语言知识应用于下游任务,获得了较好的效果~\cite{cui2021pre,houlsby2019parameter,hu2018few}。

尽管深度学习极大地推动了LJP的发展,但即便是针对法律领域优化的PLM,其能力在与新兴的大语言模型(LLM)的对比中仍显不足。

当前,以LLM为核心的技术范式成为研究热点。其强大的常识推理和零样本/少样本学习能力为LJP带来了新的可能性~\cite{brown2020language,huang2022towards}。优化思路主要分为两条路径:一是通过法律提示工程(Legal Prompt Engineering)引导模型,例如利用“思维链”(Chain-of-Thought)~\cite{kojima2022large,izacard2021leveraging}或司法三段论~\cite{huang2023lawyer}来规范其推理逻辑;二是以法律专业数据对LLM进行微调,如Lawyer LLaMA ~\cite{chen2020recall},使其具备更强的“法律素养”。

然而,将LLM应用于严肃的法律领域仍面临严峻挑战。
未经充分优化的通用LLM在处理法律问题时,容易出现“幻觉”(Hallucination)~\cite{cui2023survey},例如捏造不存在的法律条文、引用错误的案例,或给出超出法定范围的量刑建议,这在司法实践中是不可接受的~\cite{lewis2020retrieval}。
此外,尽管微调是提升专业性的有效途径~\cite{hu2021lora,hu2022lora},但它代价高昂。首先,微调的成功依赖于高质量标注数据和较大的计算资源。其次,微调后的LLM知识体系是静态的,其知识停留在训练数据集的时间点,无法实时跟进法律法规的更新与司法解释的演进~\cite{li2021prefix}。最后,微调过程还可能导致模型对其通用知识的“灾难性遗忘”~\cite{chen2020recall},损害其基础推理能力。
提示工程的局限性:为特定任务精心设计的提示词(Prompt)往往缺乏通用性,难以直接迁移至其他法律任务,这限制了该方法的可扩展性。



\section{\heiti 方法}
\subsection{\heiti 总体流程}
% \begin{figure*}[htpb]
% 	\centering
% 	\includegraphics[width=1\textwidth]{fig/method.pdf}
% 	\caption{基于大语言模型,法律引导的的案例融合方法,用于司法判决预测流程}
% 	\label{fig:main}
% \end{figure*}

本研究提出的司法判决预测方法核心在于构建一个能够有效整合案件事实、法律法规及类案判例的智能推理框架。整体流程如图\ref{fig:main}所示。
首先,系统接收待判决案件的详细事实描述($C$),并利用预训练的LLM进行初步语义分析和关键信息抽取,从复杂的案情描述中识别并初步推断出罪名类别、犯罪构成要件(包括主体、主观、客体、客观)及证据特征等核心法律要素,形成结构化的犯罪核心要素表示($F$)。
其次,基于抽取的犯罪核心要素$F$,系统并行地从法律条文数据库和案例数据库中检索相关信息。一方面,从权威的法律条文数据库中检索出与案件特征$F$高度相关的法律法规条款集合($L$)。另一方面,从海量的历史案例数据库中智能检索出与当前案件在罪名构成、事实情节和证据方面最为相似的判例集合($S$)。
最后,将原始案件事实描述$C$、初步提取的犯罪核心要素$F$、检索到的相关法律条文$L$以及筛选出的相似判例$S$共同作为上下文信息,输入至LLM,由其进行综合分析与推理,输出最终的判决结果($J$)。
该模型作为核心推理引擎,通过整合这些多源异构信息,综合考量法律原则、司法解释及类案判例的指导作用,有效弥补了 LLM 在法律专业知识和复杂逻辑推理方面的固有局限性,并且显著增强判决预测的专业性、准确性和可解释性。

\subsection{\heiti 判决核心要素提取}

在大陆法系的刑法理论中,一项行为要被判定为犯罪,其客观事实必须符合刑法分则具体条文所规定的全部构成要件。此外,法官在进行司法判决时,还需要根据检查机关提供的证据,犯罪嫌疑人的行为动机,造成的事实性后果等因素综合判决。然而,传统方法未能识别法律要素间的逻辑依赖,导致判决预测缺少法律推理的前提,最终造成判决结果的不可信甚至是判决错误~\cite{JSJA202505027,zhao2022charge,zhao2022charge}。
为了解决该问题,本研究提出了判决核心要素提取方法。判决核心要素$F$由犯罪核心要素和证据核心要素。其中犯罪要素包括犯罪主体,犯罪客体,犯罪主观,犯罪客观四个方面;证据核心要素由证据和证明力组成,证据是检查机关提供的直接证据,证明力是证据的可信程度,表明证据之间的相互印证,逻辑连贯,能够充分证明犯罪行为。该证据特征由LLM通过特别定制的。通过首先明确识别出判决核心要素, 去除冗余的信息和噪声,为后续的法律条文检索与相似案例检索提供精确,简要的数据,并且为后续的逻辑关联分析和推理奠定基础。
% 犯罪主体指实施危害社会行为并依法应当承担刑事责任的自然人或单位。例如,行为人的年龄、刑事责任能力、特殊身份(如国家工作人员)等。犯罪客体指我国刑法所保护的,而为犯罪行为所侵犯的社会主义社会关系。例如,人身权利、财产权利、国家安全、公共秩序等。犯罪主观指犯罪主体对其行为及其危害结果所持的心理态度,主要包括故意(直接故意、间接故意)和过失(疏忽大意的过失、过于自信的过失),以及特定的犯罪目的或动机。犯罪客观指犯罪活动的客观外在表现。主要包括危害行为(作为、不作为)、危害结果、行为与结果之间的因果关系,以及犯罪的时间、地点、方法、手段和某些特定情节(如“数额较大”、“多次作案”、“情节恶劣”等)。
\textit{判决核心要素提示词}从事实描述$C$中提取,如图\ref{fig:prompt1}所示。

通过明确提取犯罪核心要素和证据核心要素,不仅优化了案件结构化理解的基础,也为后续的法律条文检索、司法三段论构建及判决文本生成提供了更为精准和逻辑严谨的输入,从而提升整个智能判决系统的透明度、准确性和与人类司法实践的一致性。
\begin{figure}[H]
	\centering
	\includegraphics[width=1\linewidth]{fig/prompt1.2.pdf}
	\caption{判决核心要素提示词}
	\label{fig:prompt1}
\end{figure}
\subsection{\heiti 相关法律条文检索}

为了解决其在法律知识方面的不足和潜在的“幻觉”问题,相关法律条文检索通过引入真实的法律条文为LLM提供准确、权威的法律规范依据,确保模型在进行判决预测时,能够明确犯罪的构成要件、法律定义以及量刑的法定边界。首先构建一个包含各类法律法规的法律条文数据库($DB_{law}$​)。
\begin{figure}[H]
	\centering
	\includegraphics[width=1\linewidth]{fig/prompt2.2.pdf}
	\caption{案例数据融合提示词
	}
	\label{fig:prompt2}
\end{figure}
数据库中的每条法律条文均通过文本嵌入模型转换为高维向量表示,并存入向量数据库以支持高效检索。当处理新的案件时,从案件事实描述$C$中提取的犯罪核心要素$F$同样被转换为查询向量$F_{e}$。随后,系统采用近似最近邻搜索算法,通过计算查询向量与数据库中法律条文向量之间的相似度,检索出与案件最为相关的$k$条法律条文,形成集合$L$。该过程可表示为:
\begin{equation}
	\begin{aligned}
		L  =(l_1​,l_2​,\dots,l_k​)
		=\text{Topk}​(\text{Sim}(F_{e},DB_{law})),
	\end{aligned}
	\label{eq:L}
\end{equation}
其中,$\text{Topk}​$表示取相似度最高的$k$个结果,$\text{embed}$表示文本嵌入函数,$\text{Sim}$是相似度计算函数。
\subsection{\heiti 相似案例检索}
为了解决仅凭法条和案例描述难以覆盖所有复杂情况的问题,并促进“同案同判”, 本研究提出相似案例检索,在从历史判例中寻找与当前待审案件在核心特征上相似的案例,为LLM提供司法实践层面的参考。相似案例检索可以让LLM理解法律条文在具体情境下的应用方式,学习既往判决中蕴含的裁判经验和量刑酌情考量。

首先,构建一个结构化的案例数据库。该数据库中的每个案例都包含详细的字段,如“罪名类别”、“犯罪构成”,量刑情节”、“证据特征”、“法律适用”、“裁判逻辑”和“判决结果”。对于案例中的关键文本字段,特别是“罪名类别”、“犯罪构成”和“证据特征”,采用文本嵌入模型将其转换为向量表示,并构建相应的向量索引。需要检索相似案例时,针对当前案件提取的犯罪核心要素F中的对应字段(令检索字段集合为 $R=\{\text{罪名},\text{构成},\text{证据}\}$),即$F_{i}$​(其中 $i \in R$),分别将其通过相同的文本嵌入模型转换为查询向量$F_{i_e}$。接着,对每一个查询向量,在案例数据库对应字段的向量索引中,利用近似最近邻搜索算法和相似度计算,各自独立检索出$m$个最相似的案例。
为了得到与当前案件整体最为匹配的案例,需要对上述各字段检索出的候选案例进行综合评估。对于每一个候选案例$s_j$​,计算其与当前案件在R中所有字段上的平均相似度
{
\small
\begin{equation}
    \text{AvgSim}(s_j) = \frac{1}{|R|} \sum_{i \in R} \text{Sim}\left( F_{i_e}, \text{embed}(s_j, i) \right),
\end{equation}
}
其中,$\text{embed}(s_j,i)$表示候选案例$s_j$​在特定字段$\text{i}$上的嵌入向量。最后,选取平均相似度最高的$n$个案例$(n<m)$作为最终的相似案例集合$S$
\begin{equation}
	S=(s_1​,s_2​,\dots,s_n​)=\text{Topn​}(AvgSim(s_j​))
\end{equation}

通过引入与当前案件高度相似的历史判例,为LLM提供了宝贵的经验性知识。这不仅有助于模型更准确地把握特定罪名的构成要件和量刑尺度,还能使判决建议更符合司法实践,增强判决结果的合理性和可接受性。

\subsection{\heiti 法律约束的类案融合判决}

为了整合前述模块获取的全部信息,有效融合多源异构信息,并进行复杂法律推理的问题, 法律约束的类案融合判决将原始案件事实描述$C$、LLM从$C$中提取的结构化犯罪核心要素$F$、从法律条文数据库中检索到的相关法律条文集合$L$,以及从案例数据库中检索到的相似案例集合$S$,共同组织成一个全面的上下文\textit{案例数据融合提示词},如图\ref{fig:prompt2}所示。
这个提示被输入到预训练的LLM中。LLM利用其强大的自然语言理解、知识整合和逻辑推理能力,对这些输入信息进行深度分析和融合。模型在推理过程中,会考量法律条文L的规定(作为法律依据),并参考相似案例S中的裁判思路和判决结果(作为实践经验)。最终,LLM生成结构化的判决结果J,其内容通常包括建议的罪名、刑期、是否适用死刑或无期徒刑、相关的法律条文编号以及可能的罚金等。该过程可以概念化地表示公式:
\begin{equation}
	J=\text{LLM}(C|F,S,L),
\end{equation}
其中,$C | F,S,L $表示以$F,S,L$为条件上下文信息,结合原始描述$C$进行判决。

法律约束的类案融合判决通过整合多源数据和LLM的综合推理,实现了对案件事实、法律规范和司法判例的有效融合。它不仅提升了判决预测的准确性和专业性,还通过结合明确的法律条文和相似案例,增强了判决结果的可解释性和说服力。

\section{\heiti 实验}

\subsection{\heiti 实验数据}
\begin{table*}[htbp]
	\centering
	\caption{ 法律判决预测结果的对比}
	\begin{tabular}{lcccccc}
		\toprule
		\textbf{模型} & \multicolumn{3}{c}{\textbf{罪名}} & \multicolumn{3}{c}{\textbf{刑期}}                                                               \\
		\cmidrule(lr){2-4} \cmidrule(lr){5-7}
		            & \textbf{精确率}                    & \textbf{召回率}                    & \textbf{F1分数} & \textbf{精确率} & \textbf{召回率} & \textbf{F1分数} \\
		\midrule
		MTL-Fusion  & 0.6861                          & 0.6911                          & 0.6886        & 0.3512       & 0.3567       & 0.3539        \\
		Lawformer   & 0.6927                          & 0.7082                          & 0.7004        & 0.3581       & 0.3629       & 0.3605        \\
		BERT        & 0.7011                          & 0.7178                          & 0.7094        & 0.4311       & 0.4308       & 0.4309        \\
		LawChatGLM  & 0.7517                          & 0.7478                          & 0.7497        & 0.4712       & 0.4671       & 0.4691        \\
		Ours        & \textbf{0.7797}                          & \textbf{0.7689}                          & \textbf{0.7743}        & \textbf{0.5578}       & \textbf{0.566}        & \textbf{0.5525}        \\
		\bottomrule
	\end{tabular}
	\label{tab:performance_comparison}
\end{table*}

实验使用开源的数据集CAIL 2018~\cite{xiao2018cail2018largescalelegaldataset},共涵盖1927685个案例,覆盖202种刑事罪名和183条刑法法规,其中训练集有1927685条数据,测试集有216829条数据;在数据集中,多被告案件法条分布往往存在长尾分布现象。如图\ref{fig:acc_dis}所示,各个法条在判决结果中的出现频率及其占比有较大的不同。在测试数据的法条分布中,占比最高的5个罪行是,盗窃,危险驾驶,故意伤害和交通肇事,其占比分别达到了20.63\%,17.17\%,10.49\%,8.29\%,7.33\%;占比最高的10种罪行占总测试数据的73.58\%。而占比最少的100种罪行只占到总数据的1.83\%。
\begin{figure}[H]
		\centering
		\includegraphics[width=1\linewidth]{fig/accusation_distribution.pdf}
		\caption{测试数据的罪名分布}
		\label{fig:acc_dis}
\end{figure}
如图\ref{fig:art_dis}所示,法条数据分布也呈现长尾趋势。占比最高的10种相关法条占测试数据中0.73\%,而占比最低的100种法条只占2.44\%。
\begin{figure}[H]
    \centering
    \includegraphics[width=1\linewidth]{fig/article_distribution.pdf}
    \caption{测试数据的法条数据分布}
    \label{fig:art_dis}
\end{figure}
\subsection{\heiti 实验设置}
本研究案例数据库利用CAIL 2018~\cite{xiao2018cail2018largescalelegaldataset}的训练集构造。为了在控制案例数据库规模的同时兼顾检索效率和案例,本研究从训练集中每种罪名最多抽取100条数据,一共构建1676个案例。
法条数据库使用中国刑法作为文本数据。
案例数据库和法条数据库的文本嵌入模型使用BAAI/BGE-m3模型~\cite{chenBGEM3EmbeddingMultiLingual2024};
向量数据库使用milvus~\cite{2022manu,2021milvus},相似度函数使用内积(Inner Product,IP),向量索引方法采用倒排索引(Inverted File,IVF),聚类中心为200个。搜索算法使用暴力搜索(Flat Search,搜索在每个聚类内部时使用相似度函数进行比较。要素提取模型采用qwen-turbo模型
),判决模型使用qwen-plus模型~\cite{qwenQwen25TechnicalReport2025}。

\subsection{\heiti 评价指标}
CAIL 2018法律判决预测中的罪名、刑期预测两项子任务,都属于多标签的多分类问题~\cite{xiao2018cail2018}。
本研究采用精度(Precision , P) 、召回率(Recall , R) 和 F1分数 3项指标来衡量模型的预测效果。
\begin{eqnarray}
		&P=\frac{\sum_{i=1}^{n}TP_{i}}{\sum_{i=1}^{n}TP_{i}+\sum_{i=1}^{n}FP_{i}}
		\\
		&R=\frac{\sum_{i=1}^{n}TP_{i}}{\sum_{i=1}^{n}TP_{i}+\sum_{i=1}^{n}FN_{i}}
		\\
		&F1=\frac{2\times P\times R}{P+R}
\end{eqnarray}
其中,$i$为分类任务中类别的种类;$TP_i$为True Positive,指被正确地划分为类别 $i$ 的样本个数;$FP_i$ 为False Positive,指实际为其他类但被分类器划分为$i$ 类的样本数;$FN_i$ 为False Negative,指实际为 $i$ 类,但是被分类器划分错误的样本数。


\subsection{\heiti 基准模型}
为检验基座LLM的性能和本研究提出的“基于法条约束与类案融合的可解释司法判决预测方法”的有效性,以及与基于域外语料训练的法律大模型能力相比的优劣,本研究进行了CAIL 2018两个子任务罪名和刑期的对比实验,对比本文模型与4种基线模型的效果差异。
基线模型的选取覆盖基于词嵌入的深度学习模型MTL-Fusion~\cite{zhuopeng-etal-2020-multi}模型、中国司法长文本文书预训练模型Lawformer~\cite{xiao2021lawformer},预训练模型BERT~\cite{fan2022multi}以及 司法数据微调与RAG结合LawChatGLM模型~\cite{JSJA202505027}。

\subsection{\heiti 实验结果与分析}


表\ref{tab:performance_comparison}的实验结果清晰地揭示了不同技术路径在法律判决预测任务上的性能差异。从传统模型MTL-Fusion、Lawformer到基于预训练的BERT,再到结合了法律知识增强的LawChatGLM,模型的性能在罪名和刑期预测上呈现出稳步提升的趋势。这证明了更强的语义理解能力和外部知识的引入是提升LJP性能的关键。然而,即便是表现最强的基准模型LawChatGLM,其虽然通过检索法律条文提升了罪名预测的准确性,但在更需酌情裁量的刑期预测上表现仍有较大提升空间(F1分数为0.4691),这表明仅依赖成文法条作为外部知识源尚不足以完全捕捉司法实践的复杂性。

本文提出的方法在所有评价指标上均取得了最优表现,尤其在刑期预测任务上实现了关键突破,F1分数达到0.5525,相较于LawChatGLM提升了8.34\%。这一显著优势的核心原因在于我们创新的“法条约束与类案融合的可解释司法判决预测方法”。该框架不仅通过检索法律条文为罪名判定提供了权威的法律依据,更关键的是,通过引入“相似案例检索”模块,为模型提供了来自真实司法实践的量刑参考。这些相似判例中蕴含的裁判经验和量刑逻辑,有效弥补了成文法条在具体刑期裁量上的模糊性,使得模型的预测更贴近真实的司法裁判思维。实验结果有力地证明,将LLM的推理能力与法律条文的规范性指引、相似案例的实践性参考进行深度融合,是构建高精度、高可靠性智能司法判决系统的有效路径。

\subsection{\heiti 案例研究}
\begin{figure*}[htpb]
    \centering
    \includegraphics[width=0.8\linewidth]{fig/case.pdf}
    \caption{案例}
    \label{fig:case}
\end{figure*}
本研究提出的方法将以一个具体的盗窃案件为例进行说明。首先,系统接收一段原始的案件事实描述,例如:“公诉机关指控,2016年3月28日20时许,被告人颜某在…盗走被害人谢某支付宝内人民币3723元”。接着,系统利用LLM对该文本进行“判决核心要素提取”,生成结构化的犯罪特征($F$),包括犯罪构成(主体、主观、客体、客观)和核心证据等。随后,进入本方法的核心——双重知识检索阶段。系统会将提取出的核心要素($F$)通过文本嵌入模型BAAI/BGE-m3转换为高维查询向量。该查询向量被用于在两个并行的路径上执行检索:第一条路径中,系统在预先向量化的法律数据库中进行近似最近邻(ANN)搜索,通过计算内积相似度,找出与案件特征最匹配的法律条文($L$),如《中华人民共和国刑法》第二百六十四条;第二条路径中,系统同样利用该查询向量,在向量化的案例数据库中检索相似判例($S$)。此处的案例检索更为精细,它会综合考量罪名、犯罪构成、证据特征等多个维度的相似度,以确保筛选出的历史判例与当前案件具有高度的可比性。最后,将原始案情($C$)、提取的核心要素($F$)、检索到的法律条文($L$)及相似案例($S$)共同作为输入,送入核心的LLM推理引擎,由其进行综合分析与推理,最终生成一个的结构化判决结果($J$):
\\
accusation = ["盗窃罪"]
\\
relevant\_articles = ["264"]
\\
punish\_of\_money = 3723
\\
imprisonment = 7
\\
death\_penalty = false
\\
life\_imprisonment = false


\input{sec/5_result.tex}

\subsubsection{参考文献}
这是参考文献示例。参考文献应遵循GB/T 7741-2015标准。引用文献1~\cite{Bohan1928},文献2~\cite{chen1980zhongguo},文献3-5~\cite{bravo1990comparative,niu2013zonghe,yuan2012lana}。

\vspace {3mm}
\zihao{5}{
\noindent \textsf{致\quad 谢}\quad \textit{ *致谢内容.* 致谢}}

\vspace {5mm}
\centerline
{\zihao{5}\textsf{参~考~文~献}}
\zihao{5-} \addtolength{\itemsep}{-1em}
\vspace {1.5mm}
\bibliographystyle{gbt7714-numerical}
\bibliography{ref.bib}
\end{multicols}

\begin{multicols}{1}
\noindent {\zihao{5}\bf{附录X}.}

{\zihao{5-}\setlength\parindent{2em}
*\textbf{附录内容}置于此处,字体为小5号宋体。附录内容包括:\textbf{详细的定理证明、公式推导、原始数据}等*}\\\\

\end{multicols}


\begin{multicols}{1}
\begin{biography}[yourphotofilename.jpg]
\noindent
\textbf{First A. Author}\ \ *计算机学报第1作者提供照片电子图片,尺寸为1寸。英文作者介绍内容包括:出生年,学位(或目前学历),职称,主要研究领域(\textbf{与中文作者介绍中的研究方向一致}).*
*字体为小5号Times New Roman*
\end{biography}

\begin{biography}[yourphotofilename.jpg]
\noindent
\textbf{Second B. Author} *英文作者介绍内容包括:出生年,学位(或目前学历),职称,主要研究领域(\textbf{与中文作者介绍中的研究方向一致})。*
*字体为小5号Times New Roman*
\end{biography}
\end{multicols}


\begin{multicols}{1}
\zihao{5}
\noindent \textbf{Background}

\zihao{5-}{
\setlength\parindent{2em}
*论文背景介绍为\textbf{英文},字体为小5号Times New Roman体*

论文后面为400单词左右的英文背景介绍。介绍的内容包括:

本文研究的问题属于哪一个领域的什么问题。该类问题目前国际上解决到什么程度。

本文将问题解决到什么程度。

课题所属的项目。

项目的意义。

本研究群体以往在这个方向上的研究成果。

本文的成果是解决大课题中的哪一部分,如果涉及863$\backslash
$973以及其项目、基金、研究计划,注意这些项目的英文名称应书写正确。}

\end{multicols}
\end{document}
